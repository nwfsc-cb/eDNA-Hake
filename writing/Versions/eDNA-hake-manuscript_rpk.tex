% Options for packages loaded elsewhere
\PassOptionsToPackage{unicode}{hyperref}
\PassOptionsToPackage{hyphens}{url}
%
\documentclass[
]{article}
\usepackage{lmodern}
\usepackage{amssymb,amsmath}
\usepackage{ifxetex,ifluatex}
\ifnum 0\ifxetex 1\fi\ifluatex 1\fi=0 % if pdftex
  \usepackage[T1]{fontenc}
  \usepackage[utf8]{inputenc}
  \usepackage{textcomp} % provide euro and other symbols
\else % if luatex or xetex
  \usepackage{unicode-math}
  \defaultfontfeatures{Scale=MatchLowercase}
  \defaultfontfeatures[\rmfamily]{Ligatures=TeX,Scale=1}
\fi
% Use upquote if available, for straight quotes in verbatim environments
\IfFileExists{upquote.sty}{\usepackage{upquote}}{}
\IfFileExists{microtype.sty}{% use microtype if available
  \usepackage[]{microtype}
  \UseMicrotypeSet[protrusion]{basicmath} % disable protrusion for tt fonts
}{}
\makeatletter
\@ifundefined{KOMAClassName}{% if non-KOMA class
  \IfFileExists{parskip.sty}{%
    \usepackage{parskip}
  }{% else
    \setlength{\parindent}{0pt}
    \setlength{\parskip}{6pt plus 2pt minus 1pt}}
}{% if KOMA class
  \KOMAoptions{parskip=half}}
\makeatother
\usepackage{xcolor}
\IfFileExists{xurl.sty}{\usepackage{xurl}}{} % add URL line breaks if available
\IfFileExists{bookmark.sty}{\usepackage{bookmark}}{\usepackage{hyperref}}
\hypersetup{
  hidelinks,
  pdfcreator={LaTeX via pandoc}}
\urlstyle{same} % disable monospaced font for URLs
\usepackage[margin=1in]{geometry}
\usepackage{graphicx,grffile}
\makeatletter
\def\maxwidth{\ifdim\Gin@nat@width>\linewidth\linewidth\else\Gin@nat@width\fi}
\def\maxheight{\ifdim\Gin@nat@height>\textheight\textheight\else\Gin@nat@height\fi}
\makeatother
% Scale images if necessary, so that they will not overflow the page
% margins by default, and it is still possible to overwrite the defaults
% using explicit options in \includegraphics[width, height, ...]{}
\setkeys{Gin}{width=\maxwidth,height=\maxheight,keepaspectratio}
% Set default figure placement to htbp
\makeatletter
\def\fps@figure{htbp}
\makeatother
\setlength{\emergencystretch}{3em} % prevent overfull lines
\providecommand{\tightlist}{%
  \setlength{\itemsep}{0pt}\setlength{\parskip}{0pt}}
\setcounter{secnumdepth}{-\maxdimen} % remove section numbering
\usepackage{lineno}
\linenumbers
\usepackage{setspace}\singlespacing

\author{}
\date{\vspace{-2.5em}}

\begin{document}

\hypertarget{environmental-dna-provides-quantitative-estimates-of-abundance-and-distribution-in-support-of-fisheries-management.}{%
\subsection{Environmental DNA provides quantitative estimates of
abundance and distribution in support of fisheries
management.}\label{environmental-dna-provides-quantitative-estimates-of-abundance-and-distribution-in-support-of-fisheries-management.}}

Authors\ldots{} in no particular order at present. Andrew Olaf
Shelton\(^{1*}\), Linda Park\(^1\), Krista Nichols\(^1\), Ana
Ramon-Laca, Abigail Wells\(^2\), Sandra L.
Parker-Stetter\(^{3\dagger}\), Rebecca Thomas\(^{3}\), Chu\(^{3}\),
Julia\(^{3}\), Ryan P. Kelly\(^{4}\), Jameal F. Samhouri\(^{1}\), Blake
E. Feist\(^1\)

\(^1\)Conservation Biology Division, Northwest Fisheries Science Center,
National Marine Fisheries Service, National Oceanic and Atmospheric
Administration, 2725 Montlake Blvd. E, Seattle, WA 98112, U.S.A.

\(^2\) Lynker Associates, Under contract to Northwest Fisheries Science
Center, National Marine Fisheries Service, National Oceanic and
Atmospheric Administration, 2725 Montlake Blvd. E, Seattle, WA 98112,
U.S.A.

\(^{3}\)Fisheries Resource Analysis and Monitoring Division, Northwest
Fisheries Science Center, National Marine Fisheries Service, National
Oceanic and Atmospheric Administration, 2725 Montlake Blvd. E, Seattle,
WA 98112, U.S.A.

\(^{4}\) University of Washington, School of Marine and Environmental
Affairs, 3707 Brooklyn Ave NE, Seattle, WA 98105, U.S.A.

\(^{\dagger}\) AFSC

\(^{*}\) corresponding author:
\href{mailto:ole.shelton@noaa.gov}{\nolinkurl{ole.shelton@noaa.gov}}

\emph{Keywords}:

\pagebreak

PNAS requirements:

Research reports describe the results of original research of
exceptional importance. The preferred length of these articles is 6
pages, but PNAS allows articles up to a maximum of 12 pages. A standard
6-page article is approximately 4,000 words, 50 references, and 4
medium-size graphical elements (i.e., figures and tables).

Primary research Article

\pagebreak

\hypertarget{abstract}{%
\subsection{Abstract}\label{abstract}}

All creatures inevitably leave genetic traces in their environments, and
the resulting environmental DNA (eDNA) therefore reflects the species
present in a given habitat. It remains unclear, however, whether eDNA
signals are sufficiently quantitative for use in regulatory or policy
decisions on which human livelihoods or conservation successes may
depend. Here, we report the results of the largest eDNA ocean survey to
date (spanning 86,000 km\(^2\) to depths of 500m) to understand Pacific
hake (\emph{Merluccius productus}), the target of the largest finfish
fishery along the west coast of the United States. We sampled eDNA in
parallel with traditional acoustic survey methods and show how eDNA
provides a spatially smooth signature of hake relative to the patterns
seen in traditional acoustic survey methods. Despite local differences,
when aggregated to management relevant scales the two methods yield
comparable information about the broad-scale spatial distribution and
abundance of hake. This occurs despite eDNA arising from a limited
number of discrete samples within the larger acoustic survey. The
analysis also yields novel information about depth-specific spatial
patterns of eDNA at a large spatial scale with strong depth-specific
patterns in eDNA abundance and variability. We demonstrate the potential
power and efficacy of eDNA sampling for applied problems and posit that
eDNA methods have general quantitative applications that will prove
especially valuable in data- or resource-limited contexts.

(1)

\pagebreak

\hypertarget{introduction}{%
\subsection{Introduction}\label{introduction}}

Environmental DNA, in which the DNA from target organisms is collected
from an environmental medium (e.g.~soil or water), can detect species in
a wide range of terrestrial, aquatic, and marine habitats (2) . As a
result, eDNA may efficiently survey species diversity and changes in
community membership {[}{]} REFs. However, most applied natural resource
questions depend upon estimates of abundance (e.g.~fisheries or managing
species of conservation concern) and for these topics, eDNA must provide
information about abundance in order to be useful. While most studies
find a positive relationship between eDNA concentrations and other
survey methods (3), uncertainty about the strength of the eDNA-abundance
relationship due to the complexity of eDNA generation, transport,
degredation, and detection have limited the application of eDNA in many
quantitative applications (4). While the use of eDNA methods has
skyrocketed, growing XXX\% over the past Y years (3), reflecting
widespread adoption of eDNA technologies, basic questions about the
behavior of eDNA limit its practical application and slow its adoption
in environmental management.

Rigorous, well designed surveys underlie the successful management and
conservation of wild populations. But field surveys are expensive --
open oceans surveys involve ship time costing tens of thousands of
dollars per day, for example -- and are typically tailored to a single
or relatively narrow suite of species. eDNA methods are appealing for
open-ocean or other difficult-to-sample locations because sampling can
be fast, standardized, and non-lethal for many species simultaneously;
sampling involves only the collection and processing of environmental
samples. Even modest improvements in sampling efficiency from current
surveys can reduce the duration of surveys potentially yielding
substantial cost savings for focal species surveys and freeing survey
time to be reallocated to other understudied communities. However, much
of this kind of implementation depends upon providing eDNA-based
estimates of abundance at management-relevant scales (5--7).

Observations of eDNA differ in varying degrees from observations that
use traditional methods (e.g.~visual (8, 9), capture (10, 11), or
acoustic (7) surveys) and the degree of agreement between individual
samples of eDNA and traditional methods collected simultaneously often
determines whether eDNA is viewed as successful or a failure (7, 11).
However, eDNA observations arise from fundamentally different processes
than observations from these traditional survey methods -- most
dramatically, by exponential amplification of DNA molecules in an
environmental sample, but also because the distribution of eDNA itself
in the environment differs significantly from the distribution of its
source organisms. In the case of microbial eDNA, this distributional
distinction is negligible, but for animals of management relevance --
such as pelagic fishes -- it is not. Conceptually, fish are discrete,
while the genetic traces they leave in the water are continuous,
smoothing their environmental fingerprint over space and time.

If two methods are sampling different phenomena over different spatial
or temporal scales, we expect individual observations from those methods
to differ. For example, acoustic trawls used in fisheries stock
assessments reflect the highly patchy density of schooling fishes in
space and time. By comparison, we might expect the associated eDNA to be
distributed more evenly as a result of the lag between shedding and
decay processes. Understanding the ecology of eDNA (12) makes possible
an honest assessment of the potential uses and limitations of eDNA for
applied environmental problems, and lets us use each data stream to its
best advantage.

Here, we leverage the most spatially extensive eDNA survey of the oceans
to date -- spanning over 86,000 km\(^2\) across 10 degrees of latitude,
an area of ocean approximately equivalent to the land area Portugal, and
to depths of 500m -- to document the empirical patterns of eDNA for a
commercially important and abundant species, Pacific hake
\emph{Merluccius productus}. Hake is a semi-pelagic schooling species
and is among the most abundant species in the California Current
Ecosystem (13, 14), supporting a large and important fishery along the
Pacific coasts of US and Canada with coastwide catches in excess of
400,000t annually from 2017 to 2019 (14). The rich datasets available
for hake provide an opportunity to compare available information from
traditional surveys with eDNA.

We demonstrate the presence of large-scale, depth-specific spatial
patterns of hake DNA in the ocean ocean using a quantitative PCR assay.
We then show how eDNA can be aggregated to provide a depth-integrated
index of hake abundance comparable to acoustic-trawl survey results used
for commercial stock assessments. Consistent with the different expected
distributions of eDNA and acoustic-trawl data, the two indices are only
modestly correlated at local scales (tens of \(km^2\)) but very strongly
correlated when aggregated to managment-relevant scales (thousands of
km\(^2\)). We derive metrics of the species' spatial distribution
consistent with acoustics results, and find that eDNA provides nearly
identical model precision as the acoustic data with {[}a fraction of the
data \textbar{} discrete bottle samples, etc{]}.

\clearpage

EXTRA JUNK EXTRA JUNK EXTRA JUNK.

We and that eDNA has several appealing statistical characteristics of

(9) SALMON

(5) PNAS Using river distribution models to estimate abundance (SPATIAL
STUFF)

people who have made rinky-dink joint models: (15)

(4) Review of eDNA processes (production, transport, etc.)

(10) used gillnets in combination with metabarcoding, lakes.

(11) Metabarcoding and trawls. New Jersey.

Here we show how statistical methods from the fields of ecology and
fisheries to bear on eDNA data and yield inference about the abundance
and distribution of a commercially important fish, Pacific hake
\emph{Merc\ldots{}}. We focus on hake because the rich datasets
available for this species provided an opportunity to compare and
contrast available information with eDNA. We pair the largest eDNA
survey of the coastal ocean to date (spanning \ldots{} ) to traditional
acoustic-trawl methods conducted in parallel with eDNA samples to
understand the

We show how eDNA can simultaneously differ substantially from acoustics
estimates at local scales (kms) and yet provide very similar patterns of
abundace at large spatial scales.

In addition, we illustrate how the patterns of hake eDNA vary
substantially spatially and by depth, revealing new insights into the
ecology of eDNA. Specifically, we show how small scale variability in
eDNA appears to decline with depth, and does not follow

Our work demonstrates that there is certainly a complex and interesting
ecology of eDNA that needs additional study. While this complexity
obscures the relationships

eDNA has the potential to provide information simultaneously on a wide
range of species including many that are currently unstudied.

resulting in species of economic importance, exceptional charisma, or
severe conservation concern, being targeted for surveys while most
species are ignored. In the context of a changing climate, this bias in
surveys leaves us blind to shifting distributions\ldots{} other things.

A correlary to this fact is that we are treating eDNA samples as
conceptually identical

More importantly, the question of what can be reasonably inferred from a
given eDNA data set

Birds (Christmas counts), mammals (), fisheries In the oceans,

Integrated models are widely used in fisheries applications

By analogy deep understanding of the ``ecology" of eDNA

While the promise of eDNA has been widely described and associated
laboratory methods have been widely reviewed, case greatly democratise
which species can be surveyed as

sight unseen detection

\begin{itemize}
\tightlist
\item
  Behavior of dna in the world
\item
  Depth stratification - unveiling a third dimension
\item
  Comparison acoustics to dna
\item
  Interpretation depends on spatial scale; box of interest
\item
  Transboundary; why matters for rules
\item
  Rarefaction - how little sample could we do and get same answer?
\end{itemize}

however, in most field applications that have relevance to management or
conservation, the scale of interest is not the scale of an individual
sample but how multiple samples can be combined to infer the status and
trend of large units - a population, a meta-population, an evolutionary
significant units, etc. --

Things that are interesting.

\begin{itemize}
\tightlist
\item
  3D plot of hake by depth.
\item
  methods of eDNA give and explicit estimate of detection thresholds.
\item
  Comparison acoustics to dna
\item
  Interpretation depends on spatial scale; box of interest
\item
  Transboundary; why matters for rules
\item
  Rarefaction - how little sample could we do and get same answer?
\end{itemize}

\hypertarget{results}{%
\subsection{Results}\label{results}}

We provide the depth-specific distributions of hake eDNA over 10 degrees
of coastal ocean. Patterns of mean hake DNA concentration are distinct
with highest concentration observed between 100 and 300m depths along
the continental shelf break and south of the Oregon-California border at
42\(^{\circ}\)N. Concentrations at 500m were generally low and showed
limited spatial variation while the near surface layers (3m, 50m) showed
generally higher concentrations near the coast and in more northerly
parts of the range (Figs. 1a, 1b). There was also a large difference in
the uncertainty around DNA concentration near the surface; among the
predicted grid cells, the median coefficient of variation (CV) was
larger than 1 for both 3m and 50m but only about 0.2 for depths 100m and
deeper (Fig. 1g). Such uncertainty large observed differences DNA
concentration between replicate samples taken at the same sample
location and substantial differences among proximate sampling locations.
Together, hake DNA concentration was far more predictable at depth than
near the surface.

We combined DNA information between 50 and 500m to produce a
depth-integrated estimate of hake DNA concentration comparable to the
acoustic-trawl survey results (Fig. \ref{fig:surface.compare}). The eDNA
index showed strong spatial patterning with highest values along the
continental shelf break. In comparison results acoustic-trawl were more
spatially variable, with some areas of very high hake density and others
with very low density. At the scale of individual 5km grid cells, eDNA
and acoustic surveys were positively correlated (\(\rho\) = 0.488,
Pearson product-moment correlation) but there is considerable scatter in
the relationship. High eDNA values never occurred at locations which had
very low acoustic biomass, but very high acoustic estimates corresponded
to moderate values of eDNA. Notably, across all 3,455 grid cells,
acoustic biomass estimates had a very right-skewed distribution -- most
values were near zero with very few high values -- while eDNA values
were decidedly less skewed (Fig. \ref{fig:surface.compare}).

When aggregated to larger spatial scales, the correlation between eDNA
and acoustics increased substantially (\(\rho\) = 0.821; Fig.
\ref{fig:surface.compare}) with acoustics and eDNA scaling approximately
linearly. Such increased correlation is not dependant upon the one
degree breaks in Fig. \ref{fig:surface.compare} (see Supplement SX for
alternate spatial groupings) and shows that in terms of total biomass,
eDNA and acoustic-trawls are providing nearly equivalent information
about relative spatial abundance. At a coast-wide scale, the CV of the
acoustic-trawl estimate and eDNA index were both 0.096 . This similarity
in CV occurred despite the eDNA only being collected at 186 point
stations whereas the acoustic-trawl data includes 4,841 individual
acoustic segments and 45 mid-water trawls to determine age- and
length-structure of the hake.

Within the projection bounds of this survey, we show very similar
estimates of spatial distribution. The methods produced nearly identical
estimates of the center of gravity (median of the distributed biomass)
and very similar cumulative distributions (Fig. \ref{fig:COG}). The
\(90\%\)CI overlapping for the two methos for the entire latitudinal
range sampled.

\clearpage

\hypertarget{discussion}{%
\subsection{Discussion}\label{discussion}}

\begin{itemize}
\item
  Building spatialstatistical model for eDNA is new, methods comparable
  to acoustics and to trawl surveys
\item
  Replication of DNA allows partitioning of uncertainty ways that are
  largely impossible in other methods.
\item
  DNA is derived from many fewer observations, but provides comparable
  answers at management scales
\item
  Comparable answers between do not occur at the local level.
\item
  DNA is a wholly independent measure of abundance derived from the same
  amount of ship time.
\item
  DNA is not in the units of biomass\ldots. which hinders direct
  interpretation.
\item
  Many metrics of abundance are like this\ldots{} (e.g.~CPUE)
\item
  Acoustic and eDNA provide very similar CVs even though eDNA is
\item
  Need to think about age-structure, other sources of biomass and eDNA.
\end{itemize}

Acoustic and eDNA are independently derived estimates. They are both
related to the true abundance of fish, but no information from the
acoustics informs eDNA or vice versa. The application of eDNA, has, in
effect doubled your information about abundance without expanding the
amount of ship-time at all.

There have been some rinky-dink attempts to make joint models for eDNA
and other methods {[}(15); fukaya2020estimating{]} but they make the
mistake of assuming that the area sampled by each method is equivalent
(eDNA,individual

RANDO PARAGRAPH: Despite this debate, most papers that use eDNA
implicitly assume that eDNA accurately reflects abundance of the
targeted community; all analyses that calculate common diversity or
community metrics (e.g.~Shannon diversity, Simpson diveristy,
Bray-Curtis divergence) from eDNA are making the assumption that
observed eDNA accurately reflects the true or at least the relative
abundance of species within the sampled community. It has been well
documented that this will rarely be true when using (Kelly et al.~2019,
Amy's paper, several other.)

\clearpage

\hypertarget{methods}{%
\subsection{Methods}\label{methods}}

\hypertarget{hake-biology-summary.}{%
\subsubsection{Hake biology summary.}\label{hake-biology-summary.}}

cite Mike Malick, cite stock assessment, cite Martin Dorn,

\hypertarget{field-sampling-and-sample-processing}{%
\subsubsection{Field sampling and sample
processing}\label{field-sampling-and-sample-processing}}

We collected eDNA samples during the 2019 Joint US-Canada Acoustic trawl
survey (16). We collected water from up to six depths (3, 50, 100, 150,
300, and 500m) at 186 CTD stations spread across 36 acoustic transects
(Fig. \ref{fig:surface.compare}). We included 1769 individual 2.5L water
samples collected at 892 depth-station combinations (a small number of
samples were contaminated or lost during processing). 710 depth-staions
were collected at 50m deep or deeper using Niskin bottles. For each
Niskin sampled station, two 10L bottles were closed and after CTD
retrieval, 2.5L was collected from each bottle. Water samples from 3m
were collected from the ship's salt water intake line (MORE DETAILS). In
addition we include 49 control samples collected ship board (SEE THIS
SECTION) and performed over 6,000 quantitative PCR reactions to produce
an unprecedented description of hake DNA in the coastal ocean. All CTD
casts and therefore water collection for eDNA occurred at night.
Detailed water sampling and processing protocols can be found HERE
(SUPPLEMENT? OTHER PAPER?)

In parallel with water collection, we incorporate data on hake from the
Joint US-Canada acoustic-trawl survey (16). We include data from 4483 km
of acoustics transects representing 57 acoustic transects. 45 midwater
trawls were deployed that provide information on the age, size and
therefore signal strength of hake (14, 16). We use derived estimates of
biomass concentration (\(mt\) \(km^{-2}\)) that integrate the biomass in
the water column between depths of 50 and 500m in all analyses. All
acoustic data and associated trawls were collected during daylight
hours. Therefore there is a lag between collection of acoustic and eDNA
data, though for nearly all cases collection were separated by less than
24 hours. Methods for converting raw acoustic and trawl data to biomass
concentrations can be found in (13, 14) and references therein.

\hypertarget{water-filtration-preservation-and-dna-extraction}{%
\paragraph{Water filtration, preservation, and DNA
extraction}\label{water-filtration-preservation-and-dna-extraction}}

Overall, we follow the methods developed by and described in Ramon-Laca
et al In review. We briefly describe the sample processing steps but
refer readers to Ramon-Laca et al.~(2021) for a detailed description of
methods.

\hypertarget{quantitative-pcr-methods}{%
\paragraph{Quantitative PCR methods}\label{quantitative-pcr-methods}}

Include stuff on wash error here. How many samples, how we partitioned
some sampled sites to experimentally wash one sample and not the other.

\hypertarget{controls-and-contamination}{%
\subsubsection{Controls and
contamination}\label{controls-and-contamination}}

We found very low levels of contamination in control samples. We present
examination of the magnitude and consequences of contamination in
supplement S2.

\hypertarget{spatial-edna-model}{%
\subsubsection{Spatial eDNA Model}\label{spatial-edna-model}}

We developed a Bayesian state-space framework for modeling DNA
concentration in the coastal ocean. State-space models separate the true
biological process from the methods used to observe the process
(see\ldots{} )

We use a relatively simple process model. Let \(D_{xyd}\) be the true,
but unobserved concentration of hake DNA (DNA \(copies\) \(L^{-1}\))
present at spatial coordinates \(\{x,y\}\) (northings and eastings,
respectively, in km) and sample depth \(d\) (meters). We model the DNA
concentration as a spatially smooth process at each depth sampled
(\(d =\) 3, 50, 100, 150, 300, or 500m) and linear on the \(\log_{10}\)
scale,

\begin{align}
  & \log_{10}D_{xyd}=\gamma_d + s(b) + t_d(x,y)
\end{align}

where \(\gamma_d\) is the spatial intercept for each depth, \(s(b)\)
indicates a smoothing spline of as a function of bottom depth in meters
(\(b\)), and \(t_d(x,y)\) is a tensor-product smooth that provides an
independent spatial smooth for each depth. We use cubic regression
splines for both univariate and tensor-product smoothes.

From this process model, we construct a multi-level observation model.
First, we model the DNA concentration in each Niskin bottle \(i\), as a
random deviation from the true DNA concentration at that depth and
location and include three offsets to account for variation in the
processing of eDNA extracted from Niskin bottles. \begin{align}
  \log_{10}E_{i} &= \log_{10}D_{xyb} + \delta_i + \mathbf{I}\omega + \log_{10}{V_i} + \log_{10}{I_i} \\
  \delta_i & \sim Normal(0,\tau)
\end{align} where \(V_i\) is the proportion of 2.5 L filtered from
Niskin \(i\) (in nearly all cases this is 1), \(I_i\) is the known
dilution used to on sample \(i\) to eliminate PCR inhibition, and
\(\mathbf{I}\omega\) is an estimated offset for a ethanol wash error
(\(\omega\) is the estimated effect of the wash error and \(\mathbf{I}\)
is an indicator variable where \(\mathbf{I}=1\) for affected samples and
\(\mathbf{I}=0\) otherwise; see PCR methods below for additional
description of each offset).

When using qPCR, we do not directly observe eDNA concentration, we
observe the PCR cycle at which each sample can be detected (or if it was
never detected during the PCR). We use a hurdle model to account for the
fact that there is a detection threshold (the PCR cycle of amplification
is detected \(G=1\) or it is not \(G=0\)). Conditional on being
detected, we observe the PCR cycle (\(C\)) as a continuous variable that
follows a t-distribution, ADD CONDITIONING BAR FOR THESE EQUATIONS.

\begin{align}
  G_{ijr} &\sim Bernoulli(\phi_{0j}+\phi_{1j}\log_{10}E_{ij})\\
  C_{ijr} &\sim T(\nu,\beta_{0j}+\beta_{1j}\log_{10}E_{ij},\eta)
\end{align}

Here \(j\) indexes the PCR plate on which sample \(i\) and replicate
\(r\) were run. We conducted 3 PCR reactions for each \(E_i\). Note that
there are different intercept \((\phi_{0j},\beta_{0j})\) and slope
\((\phi_{1j},\beta_{1j})\) parameters for the \(j\)th PCR plate to allow
for among-plate variation. The t-distribution allows for heavier tails
than a normal distribution. We fix \(\nu = 3\), and estimate the
remaining parameters.

To calibrate the relationship between the number of DNA copies and PCR
cycle, each PCR plate has replicate samples with a known number of DNA
copies. These standards span six orders of magnitude (1 to 100,000
copies) and determine the relationship between copy number and PCR cycle
of detection. Let \(K_j\) be the known copy number in PCR plate \(j\),
then, ADD CONDITIONING BAR FOR THESE EQUATIONS. \begin{align}
  G_{jr} &\sim Bernoulli(\phi_{0j}+\phi_{1j}\log_{10}K_{j})\\
  C_{jr} &\sim Normal(\beta_{0j}+\beta_{1j}\log_{10}K_{j},\sigma)
\end{align} where for the standards we do not allow for heavy tails in
the observed PCR counts and use a normal likelihood rather than a
t-distributed likelihood.

Not that the use of standards provides an explicit model for the
detection threshold of eDNA when using qPCR and this detection threshold
can be directly incorporated to understand the ability to measure DNA
concentrations in field samples.

\hypertarget{spatial-models-for-acoustic-trawl-data}{%
\subsubsection{Spatial models for acoustic-trawl
data}\label{spatial-models-for-acoustic-trawl-data}}

In parallel with the model for qPCR data, we estimated a spatial model
for the hake biomass derived from the acoustic-trawl survey. The biomass
index created from the acoustic-trawl data for the entire survey area
(34.4\(^{\circ}\)N to 5XX\(^{\circ}\)N; ) is used in stock assessments
that determine the allowable catch and allocation of hake catch for
fleets from the United States, Canada, and Tribal Nations (14). As the
eDNA samples only cover a portion of this range (38.3\(^{\circ}\)N to
48.6\(^{\circ}\)N), we used the biomass observations from the subset of
transects on which we sampled eDNA to generate spatially smooth
estimates of biomass. Acoustic transects are divided into 0.926km
(0.5nm) segments and the biomass concentration within each segment is
used as data.

Unlike the eDNA data, age-specific biomass estimates are available only
as a biomass integrated across the entire water column (from depths of
50 to 500m). We fit a Bayesian hurdle model using a form similar to the
eDNA, modeling biomass concentration (\(F_{xy}\); units: \(mt km^{-2}\))
using two separate spatial submodels: the probability of occurrence and
a model for abundance conditional on the presence of hake. We model both
components as a function of a smooth of bottom depth and a spatial
smooth,

\begin{align}
  H_{xy} &\sim Bernoulli(logit^{-1}(\zeta_H + s_H(b) + t_H(x,y)))\\
  F_{xy} &\sim LogNormal(\zeta_F + s_F(b) + t_F(x,y)) - 0.5\kappa^2,\kappa)
\end{align}

where H\_\{xy\} is 1 if the observed biomass concentration is non-zero
and 0 otherwise.

where \(\zeta\) is the spatial intercept for each model component,
\(s(b)\) indicates a smoothing spline of as a function of bottom depth
in meters (\(b\)), and \(t(x,y)\) is a tensor-products smooth for each
model component.

\hypertarget{model-estimation}{%
\subsubsection{Model Estimation}\label{model-estimation}}

We implemented both the eDNA and acoustic-trawl models using the Stan
programing language as implemented in R (\emph{Rstan}) language. All
relevant code and data are provided in the online supplement. For the
eDNA model, we ran 4 MCMC chains using 1200 warm up and 2000 sampling
iterations. For the acoustic-trawl model, we ran 4 MCMC chains using
1200 warm up and 2000 sampling iterations.

We used traceplots and \(\hat{R}\) diagnostics to confirm convergence
(\(\hat{R} < 1.01\) for all parameters). There were no divergent
transitions in the sampling iterations. To generate matrices necessary
for estimating smoothes we used the R package \(brms\) (REF).

\hypertarget{coordinate-systems-covariates-and-spatial-predictions}{%
\subsubsection{Coordinate systems, covariates, and spatial
predictions}\label{coordinate-systems-covariates-and-spatial-predictions}}

We project both the acoustic-trawl model and the eDNA model to a shared
grid to enable direct comparisons between methods.

Reference Blake's 5km grid. We used bottom depth as a covariate. We
derived bottom depth from HERE AND THERE. To ensure comparability
between eDNA and acoustics, we project our results to a common grid
(equal area\ldots{} etc.) and between LAT and LAT. Bottom depth on this
grid is depth-integrated across the entire grid cell.

To create a spatial prediction for eDNA we take 2,000 draws from the
joint posterior and use eq. 1 to generate predictions for the centroid
of each grid cell and calculate posterior means and uncertainty bounds
among posterior draws. Importantly, as DNA concentration were the values
of interest, this projection does not include projection information
about the detailed observation process -- replicated Niskin samples from
a site for example.

Blake will write a paragraph about area-weighted mean depth used as a
covariate.

Ref for the 5im grid:

(17)

\hypertarget{creating-an-edna-index}{%
\subsubsection{Creating an eDNA index}\label{creating-an-edna-index}}

Our model provides direct predictions for hake DNA concentration at
depths of 50, 100, 150, 300, and 500m. To produce an index spanning
depths of 50 to 500m, we need to equally weight depths between 50 and
500m. As we lack observations at other depths, we used linear
interpolation between the closest depths using the posterior predictions
at each depth to provide predicted DNA densities at 200, 250, 350, 400,
and 450m for each 5km grid cell. For example, a prediction for a grid
cell at 400m would be the mean of the predicted value for that grid cell
from a single posterior draw for 300m and 500m. Because some spatial
locations have depths of less than 500m, we only include predicted DNA
concentrations to a depth appropriate for the bathymetry (e.g.~a
location with a depth of 180m only includes values from 50, 100 and
150m). We sum across all depths (between 50 and up to 500m) to generate
a depth-integrated index of hake DNA. This index will be proportional to
the hake DNA found in the water column. However, as we are only summing
across discrete depths, not integrating values across the entire water
column nor multiplying by the total water volume within each grid cell,
the absolute value of the index will depend upon the number of discrete
depths we use. As such, we refer to this as a DNA index to differentiate
it from the predictions to a specific depths which have units of copies
per liter.

\newpage
\clearpage

\hypertarget{figures}{%
\section{Figures}\label{figures}}

\begin{figure}
\includegraphics[width=1\linewidth]{./Pub_Figs/Hake maps Mean depth manual facet v2} \caption{\label{fig:mean.maps} Predicted DNA concentration for six water depths. }\label{fig:fig.mean.maps}
\end{figure}

\begin{figure}
\includegraphics[width=1\linewidth]{./Pub_Figs/Hake maps combined to surface} \caption{\label{fig:surface.compare} 2019 survey locations (left; red circles show eDNA sampling locations, lines show acoustic transects), depth-integrated  index of hake DNA (middle) and hake biomass from acoustic surveys (right).  Both DNA and acoustic estimates are mean predicted values projected to a 5km grid and include information between 50 and 500m deep. All panels show one degree latitudinal bins used to aggregate abundance estimates over larger spatial scales (see also Fig. 3).    }\label{fig:fig.surface.compare}
\end{figure}

\begin{figure}
\includegraphics[width=1\linewidth]{./Pub_Figs/Hake point-level and 1 degree} \caption{\label{fig:pairwise} Pairwise comparison between  DNA and acoustics-derived biomass. Left panel show the posterior mean prediction from each method among the 3,455 25km2 grid cells and includes the marginal histogram of posterior mean values for each method (RHO = 0.488).  Right panel showes correlation between methods among the 11, one degree bins (posterior mean, 90 CI; RHO = 0.8208). Numbers indicate associated region identified in Fig. 2.  }\label{fig:fig.pairwise}
\end{figure}

\clearpage

\begin{figure}
\includegraphics[width=1\linewidth]{./Pub_Figs/Hake compare distribution} \caption{\label{fig:COG} Estimates of distribution from acoustic and eDNA methods. left: Cumulative distribution between 38.3 and 48.6N ( posterior means and 90 CI). Right: Center of gravity (median of distribution) for each method (posterior means and 90 CI).  Only areas within the projection grid are included in this calculation (Figs. 1, 2) are included.}\label{fig:fig.COG}
\end{figure}

\clearpage

\hypertarget{citations}{%
\subsection*{Citations}\label{citations}}
\addcontentsline{toc}{subsection}{Citations}

\hypertarget{refs}{}
\leavevmode\hypertarget{ref-murakami2019dispersion}{}%
1. H. Murakami\emph{et al.}, Dispersion and degradation of environmental
dna from caged fish in a marine environment. \emph{Fisheries science}
\textbf{85}, 327--337 (2019).

\leavevmode\hypertarget{ref-thomsen2015environmental}{}%
2. P. F. Thomsen, E. Willerslev, Environmental DNA--an emerging tool in
conservation for monitoring past and present biodiversity.
\emph{Biological Conservation} \textbf{183}, 4--18 (2015).

\leavevmode\hypertarget{ref-rourke2021environmental}{}%
3. M. L. Rourke\emph{et al.}, Environmental dna (eDNA) as a tool for
assessing fish biomass: A review of approaches and future considerations
for resource surveys. \emph{Environmental DNA} (2021).

\leavevmode\hypertarget{ref-harrison2019predicting}{}%
4. J. B. Harrison, J. M. Sunday, S. M. Rogers, Predicting the fate of
eDNA in the environment and implications for studying biodiversity.
\emph{Proceedings of the Royal Society B} \textbf{286}, 20191409 (2019).

\leavevmode\hypertarget{ref-carraro2018estimating}{}%
5. L. Carraro, H. Hartikainen, J. Jokela, E. Bertuzzo, A. Rinaldo,
Estimating species distribution and abundance in river networks using
environmental dna. \emph{Proceedings of the National Academy of
Sciences} \textbf{115}, 11724--11729 (2018).

\leavevmode\hypertarget{ref-shelton2019biocons}{}%
6. A. O. Shelton\emph{et al.}, Environmental dna provides quantitative
estimates of a threatened salmon species. \emph{Biological Conservation}
\textbf{237}, 383--391 (2019).

\leavevmode\hypertarget{ref-Fukaya2020estimating}{}%
7. K. Fukaya\emph{et al.}, Estimating fish population abundance by
integrating quantitative data on environmental dna and hydrodynamic
modelling. \emph{Molecular ecology} (2020).

\leavevmode\hypertarget{ref-port2016assessing}{}%
8. J. A. Port\emph{et al.}, Assessing vertebrate biodiversity in a kelp
forest ecosystem using environmental DNA. \emph{Molecular Ecology}
\textbf{25}, 527--541 (2016).

\leavevmode\hypertarget{ref-tillotson2018concentrations}{}%
9. M. D. Tillotson\emph{et al.}, Concentrations of environmental DNA
(eDNA) reflect spawning salmon abundance at fine spatial and temporal
scales. \emph{Biological Conservation} \textbf{220}, 1--11 (2018).

\leavevmode\hypertarget{ref-hanfling2016gillnet}{}%
10. B. Hänfling\emph{et al.}, Environmental dna metabarcoding of lake
fish communities reflects long-term data from established survey
methods. \emph{Molecular ecology} \textbf{25}, 3101--3119 (2016).

\leavevmode\hypertarget{ref-stoeckle2021trawl}{}%
11. M. Y. Stoeckle\emph{et al.}, Trawl and eDNA assessment of marine
fish diversity, seasonality, and relative abundance in coastal new
jersey, usa. \emph{ICES Journal of Marine Science} \textbf{78}, 293--304
(2021).

\leavevmode\hypertarget{ref-barnes2016ecology}{}%
12. M. A. Barnes, C. R. Turner, The ecology of environmental dna and
implications for conservation genetics. \emph{Conservation genetics}
\textbf{17}, 1--17 (2016).

\leavevmode\hypertarget{ref-malick2020relationships}{}%
13. M. J. Malick\emph{et al.}, Relationships between temperature and
pacific hake distribution vary across latitude and life-history stage.
\emph{Marine Ecology Progress Series} \textbf{639}, 185--197 (2020).

\leavevmode\hypertarget{ref-grandin2020assessment}{}%
14. C. J. Grandin, K. F. Johnson, A. W. Edwards, A. M. Berger, ``Status
of the pacific hake (whiting) stock in u.s. And canadian waters in
2020.'' (Prepared by the Joint Technical Committee of the U.S.; Canada
Pacific Hake/Whiting Agreement, National Marine Fisheries Service;
Fisheries; Oceans Canada, 2020).

\leavevmode\hypertarget{ref-chambert2018analytical}{}%
15. T. Chambert, D. S. Pilliod, C. S. Goldberg, H. Doi, T. Takahara, An
analytical framework for estimating aquatic species density from
environmental dna. \emph{Ecology and evolution} \textbf{8}, 3468--3477
(2018).

\leavevmode\hypertarget{ref-deBlois2020survey}{}%
16. S. de Blois, The 2019 joint u.s.--Canada integrated ecosystem and
pacific hake acoustic-trawl survey: Cruise report sh-19-06. \emph{U.S.
Department of Commerce, NOAA Processed Report NMFS-NWFSC-PR-2020-03}
(2020).

\leavevmode\hypertarget{ref-feist2021footprints}{}%
17. B. E. Feist, J. F. Samhouri, K. A. Forney, L. E. Saez, Footprints of
fixed-gear fisheries in relation to rising whale entanglements on the us
west coast. \emph{Fisheries Management and Ecology} \textbf{28},
283--294 (2021).

\end{document}
